\documentclass[a4paper,12pt]{article}
\usepackage[finnish]{babel}
\usepackage[utf8]{inputenc}
\usepackage[pdftex]{graphicx}
\usepackage[T1]{fontenc}
\usepackage{lmodern}
\usepackage[protrusion=true,expansion=true]{microtype}
\usepackage{float}


\begin{document}
\begin{titlepage}
\title{
AS-74.2400 Systeemidynamiikka harjoitustyö}
\author{
\begin{tabular}{ l l l }
\\
  Karri Kumara & opknumero  &  sposti\\
  Ilkka Laine &  225911 &  ilkka.laine@aalto.fi \\
\end{tabular}
}
\date{\vspace{3cm} \today}
\maketitle
\title{}
\end{titlepage}



\section{Tavoite}

Määritellään tutkimusongelma, tutkittavat muuttujat, aikahorisontti ja muut rajaukset. Sitten esitellään hypoteesit.

\section{Malli}
Suurin osa raportista keskittyy itse mallintamisprosessin kuvailuun, eli mitä teitte ja miksi.


\subsection{Mallin rakenne}
ddddddd rivinvaihto \\
 d

\subsection{}


\section{Tulokset}
Lopuksi raportissa täytyy esitellä tulokset ja johtopäätökset. On tärkeää myös kertoa yksinkertaistuksista ja
oletuksista, joita mallinnuksessa on tehty. On myös hyvä selittää, miksi mallin antamat tulokset eroavat
historiadatasta ja miten mallia voisi parantaa.








%Kirjallisuuviitteet
\begin{thebibliography}{99}
% Kirjallisuus viitteet tähän tapaan:
%\item{R.W. Robinnet, Quantum Mechanics, Oxford University Press, 1997}
%\label{robinnet}
\item{viite 1}
\label{viite1}

\end{thebibliography}


%liitteet numeroituna
\section*{Liitteet}
\begin{enumerate}
%Liitteet tähän tapaan
\item{liite 1} \label{liite1}

\end{enumerate}

\end{document}
